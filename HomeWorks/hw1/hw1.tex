\documentclass{article}
\usepackage{graphicx}
\usepackage[utf8]{inputenc}
\usepackage{fullpage}
\usepackage{listings}
\usepackage{xcolor}
\usepackage{url}
\usepackage[linesnumbered,ruled,vlined]{algorithm2e}
\usepackage{enumitem}

\parindent0in
\pagestyle{plain}
\thispagestyle{plain}

\newcommand{\assignment}{Homework 1}
\newcommand{\duedate}{July 19, 2019}

% \renewcommand\thesubsection{\arabic{subsection}}

\title{Homework 1}
\date{}

\begin{document}

Fundação Getulio Vargas\hfill\\
Estruturas de Dados e Algoritmos\hfill\textbf{\assignment}\\
Prof.\ Jorge Poco\hfill\textbf{Due:}: \duedate\\
\smallskip\hrule\bigskip

{\let\newpage\relax\maketitle}
\maketitle


\section{Induction [3pts]}
Answers should be written in this document. 

\begin{enumerate}
  \item Prove by Induction that:
  \( \sum_{i=1}^{n}i^2=\frac{n(n+1)(2n+1)}{6} \qquad\forall n \geq 0\)

  \item Prove by Induction that:
  $\forall n \geq 7$ it is true $3^n<n!$
  
  \item Prove by Induction that $\forall n \geq 0$
  \[
    \left \lceil\frac{n}{2} \right \rceil=
    \left\{
    \begin{array}{ll}
    \frac{n}{2}& \textrm{si $n$ es par}\\
    \frac{n+1}{2}& \textrm{si $n$ es impar}
    \end{array}
    \right.
  \]
  
    \item Prove by induction that a number is divisible by 3 if and only if the sum of its digits is divisible by 3.
    
    \item Prove that any integer greater than 59 can be formed using only 7 and 11 cent coins.
  
  \item Prove by induction that $F_{n+k}=F_{k}F_{n+1}+F_{k-1}F_{n}$
  
  \item Prove by induction in $n$ that \(\sum_{m=0}^{n}{n \choose m}=2^n\)
  
  \item Prove by induction that a graph with $n$ vertices can have at most  $\frac{n(n-1)}{2}$ edges.
  
  \item Prove by induction that a complete binary tree\footnote{http://web.cecs.pdx.edu/~sheard/course/Cs163/Doc/FullvsComplete.html} with $n$ levels has $2^n-1$ vertices.
  
  \item A polygon is convex if each pair of points in the polygon can be joined by a straight line that does not leave the polygon. Prove by induction in $n>3$ that the sum of the angles of a polygon of $n$ vertices is $180(n-2)$.
  
\end{enumerate}

\section{Correctness of bubblesort [2pts]}
Bubblesort is a popular, but inefficient, sorting algorithm. It works by repeatedly swapping adjacent elements that are out of order.

\begin{algorithm}[H]
\SetAlgoLined
  \For{$i = 1$ \textbf{to} $A.length -1$} {
    \For{$j = A.length$ \textbf{downto} $i + 1$} {
      \If{$A[j] < A[j-1]$} {
        exchange $A[j]$ with $A[j-1]$
      }
    }
  }
\caption{BUBBLESORT(A)}
\end{algorithm}

\begin{enumerate}[label=\Alph*]
  \item Let $A'$ denote the output of BUBBLESORT(A). To prove that BUBBLESORT is correct, we need to prove that it terminates and that
  
  \begin{equation} \label{eq:1}
    A'[1] \leq A'[2] \leq ... \leq A'[n]
  \end{equation}
  
  where $n = A.length$. In order to show that BUBBLESORT actually sorts, what else do we need to prove?
  
  The next two parts will prove inequality~(\ref{eq:1}).
  
  \item State precisely a loop invariant for the \textbf{for} loop in lines 2–6, and prove that this loop invariant holds. Your proof should use the structure of the loop invariant proof presented in this chapter.
  
  \item Using the termination condition of the loop invariant proved in part (B), state a loop invariant for the for loop in lines 1–7 that will allow you to prove inequality~(\ref{eq:1}). Your proof should use the structure of the loop invariant proof presented in this chapter.
  
  \item What is the worst-case running time of BUBBLESORT? How does it compare to the running time of insertion sort?
\end{enumerate}


\section{Growth of Functions [2pts]}

\begin{enumerate}[label=\Alph*]
  \item For each of the following pairs of functions, either $f(n)$ is in $O(g(n))$, $f(n)$ is in $\Omega(g(n))$, or $f(n) = \Theta(g(n))$. Determine which relationship is correct and briefly explain why.
    \begin{itemize}
      \item $f(n) = \log n^2$; $g(n) = \log n + 5$
      \item $f(n) = \log^2 n$; $g(n) = \log n$
      \item $f(n) = n\log n + n$; $g(n) = \log n$
      \item $f(n) = 2^n$; $g(n) = 10n^2$
    \end{itemize}
  
  \item Prove that $n^3 -3n^2 -n+1 = \Theta(n^3)$.
  \item Prove that $n^2 = O(2^n)$.
  
\end{enumerate}


\section{Insertion Sort - Mergesort - Quicksort [3pts]}
Implement the insertion sort, merge sort and quicksort using the template \texttt{test.py} (use Python 3.X). Create a \texttt{test.cpp} file and write the equivalent code from \texttt{test.py} in C++, ie., the functions: main, \texttt{insertion\_sort}, \texttt{merge\_sort}, \texttt{quicksort} and \texttt{is\_sorted}. For the random number generations you can use the \texttt{rand} function from \texttt{cstdlib}\footnote{\url{http://www.cplusplus.com/reference/cstdlib/rand/}}. Your code should print the tuple (number of objects, time insertion\_sort, time merge\_sort, time quicksort)

You must submit both \texttt{test.py} and \texttt{test.cpp}. Graphs and descriptions must be included in this document. 

\subsection{Random Order}
\begin{enumerate}
  \item Create 10 sets of numbers in random order. The sets must have \{10k, 20k, 30k, ..., 100k\} numbers.
  
  \item Sort these numbers using the 3 algorithms and calculate the time each algorithm takes for each set of numbers.
  
  \item Generate a plot (using excel or another tool) showing a \emph{linechart}, where the $x$-axis is the ``number of elements", and the $y$-axis is the time that the algorithms took in C++ and Python. This plot must have 6 lines of different colors with a legend.
  
  \item Write a small paragraph (3 to 4 lines) describing the results.
\end{enumerate}

\subsection{Ascending Order}
Do the same experiment when the numbers are ordered in ascending order.

\subsection{Descending Order}
Do the same experiment when the numbers are ordered in descending order.




\end{document}