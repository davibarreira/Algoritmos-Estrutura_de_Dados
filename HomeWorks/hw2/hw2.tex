\documentclass{article}
\usepackage{graphicx}
\usepackage[utf8]{inputenc}
\usepackage{fullpage}
\usepackage{listings}
\usepackage{xcolor}
\usepackage{url}
\usepackage[linesnumbered,ruled,vlined]{algorithm2e}
\usepackage{enumitem}


\definecolor{mygreen}{rgb}{0,0.6,0}

% set the default code style
\lstset{
    language=C++,
    frame=tb, % draw a frame at the top and bottom of the code block
    tabsize=4, % tab space width
    showstringspaces=false, % don't mark spaces in strings
    numbers=left, % display line numbers on the left
    commentstyle=\color{mygreen}, % comment color
    keywordstyle=\color{blue}, % keyword color
    stringstyle=\color{red}, % string color
    backgroundcolor=\color{black!5}, % set backgroundcolor
    basicstyle=\footnotesize,% basic font setting
}

\parindent0in
\pagestyle{plain}
\thispagestyle{plain}

\newcommand{\assignment}{Homework 2}
\newcommand{\duedate}{August 13, 2019}


% \renewcommand\thesubsection{\arabic{subsection}}

\title{Homework 2}
\date{}

\begin{document}

Fundação Getulio Vargas\hfill\\
Estruturas de Dados e Algoritmos\hfill\textbf{\assignment}\\
Prof.\ Jorge Poco\hfill\textbf{Due:}: \duedate\\
\smallskip\hrule\bigskip

{\let\newpage\relax\maketitle}
\maketitle


\section{Red-Black Trees}

I am attaching a binary tree source code (\texttt{bst-0.0.cpp}) with the methods insert, delete and print. Your job would be to implement a Red-Black Tree with the functions insert, remove and print. 

To test your code you can follow the examples described in the document \texttt{anexo1.pdf}. In addition, you might be interested in the document \texttt{anexo2.pdf} for a more detailed description of this tree, there is also some Java code that might be useful. 

Note, your code must be implemented in C++ and based in the BST class I'm providing you. Grading would be as follow:

\begin{enumerate}[label=(\alph*)]
  \item \textbf{(2.5pts)} insert 
  \item \textbf{(2.5pts)} remove 
\end{enumerate}

An example of the main function is: 

\begin{lstlisting}
int main() {
  // this constructor must call the function insert multiple times 
  // respecting the order
  RBTree tree(41, 38, 31, 12, 19, 8);
  tree.print();

  // testing the remove function
  tree.remove(8);
  tree.print();
}
\end{lstlisting}




\section{Radix Sort}

\textbf{(2pts) }Your job is to implement the radix sort algorithm in Python. The following code is going to be used to test your implementation. You have to submit a notebook with your code. 
  
\begin{lstlisting}[language=Python]
def radix_sort(A, d, k):
  # A consists of n d-digit ints, with digits ranging 0 -> k-1
  #
  # implement your code here
  # return A_sorted


# Testing your function
A = [201, 10, 3, 100]
A_sorted = radix_sort(A, 3, 10)
print(A_sorted)
# output: [3, 10, 100, 201]
\end{lstlisting}

\section{Sorting in Place in Linear Time}
\textbf{(1.5pts)} Suppose that we have an array of $n$ data records to sort and that the key of each record has the value 0 or 1. An algorithm for sorting such a set of records might possess some subset of the following three desirable characteristics:

\begin{enumerate}
  \item The algorithm runs in $O(n)$ time.
  \item The algorithm is stable.
  \item The algorithm sorts in place, using no more than a constant amount of storage space in addition to the original array.
\end{enumerate}

\begin{enumerate}[label=(\alph*)]
  \item Give an algorithm that satisfies criteria 1 and 2 above.
  \item Give an algorithm that satisfies criteria 1 and 3 above.
  \item Give an algorithm that satisfies criteria 2 and 3 above.
  \item Can any of your sorting algorithms from parts(a)–(c) be used to sort $n$ records with $b$-bit keys using radix sort in $O(bn)$ time? Explain how or why not.
  \item Suppose that the $n$ records have keys in the range from 1 to $k$. Show how to modify counting sort so that the records can be sorted in place in $O(n + k)$ time. You may use $O(k)$ storage outside the input array. Is your algorithm stable? (Hint: How would you do it for $k = 3$?)

\end{enumerate}

\section{Alternative Quicksort Analysis} 
\textbf{(1.5pts)} An alternative analysis of the running time of randomized quicksort focuses on the expected running time of each individual recursive call to QUICKSORT, rather than on the number of comparisons performed.

\begin{enumerate}[label=(\alph*)]
  \item Argue that, given an array of size $n$, the probability that any particular element is chosen as the pivot is $1/n$. Use this to define indicator random variables $X_i = I \{i\mbox{th smallest element is chosen as the pivot}\}$. What is $E[X_i]$?
  \item Let $T(n)$ be a random variable denoting the running time of quicksort on an array of size $n$. Argue that
  \begin{equation}
    E[T(n)]=E\left[\sum_{q=1}^{n}X_q(T(q-1)+T(n-q)+\Theta(n))\right]  
    \label{eq:1}
  \end{equation}
  
  \item Show that equation~\ref{eq:1} simplifies to
  \begin{equation}
    E[T(n)] = \frac{2}{n}\sum_{q=0}^{n-1}E[T(q)] + \Theta(n)
    \label{eq:2}
  \end{equation}

  \item Show that
  \begin{equation}
    \sum_{k=1}^{n-1} k \lg k \leq \frac{1}{2}n^2\lg n - \frac{1}{8}n^2
    \label{eq:3}
  \end{equation}
  (Hint: Split the summation into two parts, one for $k=1,2, \ldots, \lceil n/2 \rceil - 1$ and \\ one for $k=\lceil n/2 \rceil~,\ldots,~n-1.)$

  \item Using the bound from equation~\ref{eq:3}, show that the recurrence in equation~\ref{eq:2} has the solution $E[T(n)]=\Theta(n\lg n)$. (Hint: Show, by substitution, that $E[T(n)] \leq an \log n - bn$ for some positive constants $a$ and $b$.)
\end{enumerate}



\end{document}